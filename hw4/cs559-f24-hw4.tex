\documentclass{exam}

\usepackage{amsmath}

\usepackage{amssymb}

\usepackage{graphicx}

\usepackage{cite}
\usepackage{color} 
\usepackage{setspace}
\usepackage{hyperref}
\usepackage[linewidth=1pt]{mdframed}
\usepackage{tcolorbox}
\usepackage{hyperref}
\usepackage{listings}
\usepackage{color}

\definecolor{dkgreen}{rgb}{0,0.6,0}
\definecolor{gray}{rgb}{0.5,0.5,0.5}
\definecolor{mauve}{rgb}{0.58,0,0.82}

\lstset{frame=tb,
  language=Java,
  aboveskip=3mm,
  belowskip=3mm,
  showstringspaces=false,
  columns=flexible,
  basicstyle={\small\ttfamily},
  numbers=none,
  numberstyle=\tiny\color{gray},
  keywordstyle=\color{blue},
  commentstyle=\color{dkgreen},
  stringstyle=\color{mauve},
  breaklines=true,
  breakatwhitespace=true,
  tabsize=3
}

\newcommand{\xx}{{\bf{x}}}
\newcommand{\yy}{{\bf{y}}}
\newcommand{\ww}{{\bf{w}}}

\pagestyle{headandfoot}
\runningheadrule
\firstpageheader{CS559: Machine Learning}{Name:        }{\textcolor{red}{Due: Nov. 14, 2024}}

\title{Assignment 4: SVM, Trees, and Neural Nets}
\date{}
\begin{document}
\maketitle
\thispagestyle{headandfoot}

\begin{center}
  {\fbox{\parbox{5.5in}{\centering
\textbf{Homework assignments will be done individually: each student must hand in their own answers. Use of partial or entire solutions obtained from others or online is strictly prohibited. Electronic submission on Canvas is mandatory.}}}}
\end{center}
\vspace{.5cm}


\vspace{10pt}
\noindent{\underline{\bf Submission Instructions}}: 
You shall submit a zip file named A4\_FirstName\_LastName.zip which contains:

\begin{itemize}
  \item python files (.ipynb or .py) including all the code, comments, plots, and result analysis. You need to provide detailed comments in English.
   \item pdf file including (a) solutions for questions 1 and 2, (b) descriptions of observations for questions 3,4,5. 
\end{itemize}


\begin{questions}

\begin{table}[h]
	\centering
	\caption{Data set for Question 1}\label{data-q1}
	\begin{tabular}{c|c|c|c|c}
		\hline
		data & $x_{i1}$ & $x_{i2}$ & $y_i$ & $\alpha_i$\\
			\hline
		$\xx_1$ & 4 & 2.9 & 1 & 0.414 \\
			$\xx_2$ & 4 &4 & 1 & 0 \\
				$\xx_3$ & 1 & 2.5 & -1 & 0\\
					$\xx_4$ & 2.5 & 1& -1 & 0.018 \\
						$\xx_5$ & 4.9 & 4.5& 1 & 0 \\
							$\xx_6$ & 1.9 & 1.9 & -1 & 0\\
								$\xx_7$ &3.5 & 4 & 1 & 0.018 \\
									$\xx_8$ & 0.5 & 1.5 & -1 & 0 \\
										$\xx_9$ & 2 & 2.1 & -1 & 0.414 \\
											$\xx_{10}$ & 4.5& 2.5& 1 & 0\\
			\hline
\end{tabular}
\end{table}
\question{\bf Support Vector Machines}\label{q1} (4 points) Given 10 points in Table~\ref{data-q1}, along with their classes and their Lagranian multipliers ($\alpha_i$), answer the following questions:


\begin{parts}
\part (1 pts) What is the equation of the SVM hyperplane $h(x)$? Draw  the hyperplane with the 10 points.

\part (2 pts) What is the distance of $x_6$ from the hyperplane? Write down the process of how the distance is calculated. Is it within the margin of the classifier?


\part (1 pts) Classify the point $z = (3, 3)^\top$ using $h(x)$ from above.

\end{parts}

\newpage
\question{\bf Support Vector Machines}\label{q1} (4 points) The SVM loss function with slack variables can be viewed as:
	\begin{align}
		      \min_{\mathbf{w},b} \frac{ ||\mathbf{w}||^2}{2} + \gamma \sum_{i=1}^N \underbrace{\max(0, 1- y_i f(\xx_i))}_{\textcolor{red}{\text{Hinge loss}}}
		\end{align}

The hinge loss provides a way of dealing with datasets that are not separable.

\begin{parts}
\part  (1 pts) Argue that $l = \max(0, 1-y\ww^\top\xx)$ is convex as a function of $\ww$.

\part  (1 pts) Suppose that for some $\ww$ we have a correct prediction of $f$ with $\xx_i$, i.e. $f(\xx_i) = \text{sgn}(\ww^\top \xx_i)$. For binary classifications ($y_i = +1/-1$), what range of values can the hinge loss, $l$, take on this correctly classified example? Points which are classified correctly and
which have non-zero hinge loss are referred to as margin mistakes.


\part (2 pts) Let $M(\ww)$ be the number of mistakes made by $\ww$ on our dataset (in terms of classification loss). Show that:
$$\frac{1}{n}M(\ww) \leq \frac{1}{n}\sum_{i=1}^n \max(0, 1- y_i\ww^\top \xx_i)$$

In other words, the average hinge loss on our dataset is an upper bound on the average number of mistakes we make on our dataset.
\end{parts}

\newpage
\question{\bf  Decision Trees} (4 points) Implement a Decision Tree model for the \href{https://www.kaggle.com/c/titanic/data}{Titanic} data set (only use the train file). \textbf{Do not use packages for implementing decision trees.}
\begin{itemize}
\item (1 pts) Explain how you preprocess the features. Divide the train file into a training set and a test set.
\item (2 pts) Build a tree on the training data and evaluate the classification performance on the test data.
% \item (4 pts) Try both Gini index and Information Gain as the attribute selection in building the decision tree; Compare their results on the test set.
\item (1 pts) Report your best accuracy on the test data set. Give a brief description of your observations.
\end{itemize}

\newpage
\question{\bf  Boosting} (4 points) Implement AdaBoost for the \href{https://www.kaggle.com/c/titanic/data}{Titanic} data set. You can use package/tools to implement your decision tree classifiers. The fit function of DecisionTreeClassifier in sklearn has a parameter: sample weight, which you can use to weigh training examples differently during various
rounds of AdaBoost. 
\begin{itemize}
\item (2 pts) Implement the AdaBoost algorithm.
\item (1 pts) Plot the train and test errors as a function of the number of rounds from 1 through 500. 
\item (1 pts) Report your best accuracy on the test data set. Give a brief description of your observations.
\end{itemize}

 \newpage
 \question{\bf  Neural Networks} (4 points)  Apply a Neural Network (NN) model to predict a handwritten digit images into $0$ to $9$.  The pickled file represents a tuple of 3 lists : the training set, the validation set and the test set. Each of the three lists is a pair formed from a list of images and a list of class labels for each of the images. An image is represented as numpy 1-dimensional array of 784 (28 x 28) float values between $0$ and $1$ ($0$ stands for black, $1$ for white). The labels are numbers between $0$ and $9$ indicating which digit the image represents. The code block below shows how to load the dataset.
 
 \begin{lstlisting}
import cPickle, gzip, numpy

# Load the dataset
f = gzip.open('mnist.pkl.gz', 'rb')
train_set, valid_set, test_set = cPickle.load(f)
f.close()
\end{lstlisting}

\begin{itemize}
\item (2 pts) Build a neural network with one or two hidden layers using the training data and predict the labels of data samples in the test set. Choose a loss function for this classification problem. You can use packages for neural networks. If you write a neural network from scratch with forward pass and back-propagation, you will get extra credits.
\item (1 pts) Tune your model hyper-parameters (number of hidden neurons in your hidden layers) on the validation set.
\item (0.5 pts) Plot the train, validation, and test errors as a function of the epochs (x-axis: epochs, y-axis: errors; Report the best accuracy on the validation and test data sets.
\item (0.5 pts) Apply early stopping using the validation set to avoid overfitting. Give a brief description of your observations.
\end{itemize}
\end{questions}



\end{document}