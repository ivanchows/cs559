\documentclass{article}
\usepackage{arxiv}
\setcounter{secnumdepth}{2} 
\fancyfoot{}
\sloppy
\usepackage{wrapfig}
\usepackage{booktabs} % For formal tables
\usepackage{caption}
\usepackage{subcaption}
\usepackage{amsmath,amssymb,amsfonts}
\usepackage{algorithmic}
\usepackage[numbers]{natbib}
\usepackage{comment}
\usepackage[linesnumbered,ruled,vlined]{algorithm2e}
\SetKwComment{Comment}{$\triangleright$\ }{}
\usepackage{graphicx}
\usepackage{textcomp}
\usepackage{multirow}
\usepackage{hyperref}
\usepackage[table, svgnames, dvipsnames]{xcolor}
\usepackage{makecell, cellspace, caption}
\def\BibTeX{{\rm B\kern-.05em{\sc i\kern-.025em b}\kern-.08em
    T\kern-.1667em\lower.7ex\hbox{E}\kern-.125emX}}

\def\A{{\bf A}}
\def\a{{\bf a}}
\def\B{{\bf B}}
\def\b{{\bf b}}
\def\C{{\bf C}}
\def\c{{\bf c}}
\def\D{{\bf D}}
\def\d{{\bf d}}
\def\E{{\bf E}}
\def\e{{\bf e}}
\def\F{{\bf F}}
\def\f{{\bf f}}
\def\g{{\bf g}}
\def\h{{\bf h}}
\def\G{{\bf G}}
\def\H{{\bf H}}
\def\I{{\bf I}}
\def\K{{\bf K}}
\def\k{{\bf k}}
\def\l{{\bf l}}
\def\M{{\bf M}}
\def\m{{\bf m}}
\def\N{{\bf N}}
\def\n{{\bf n}}
\def\Q{{\bf Q}}
\def\q{{\bf q}}
\def\R{{\bf R}}
\def\S{{\bf S}}
\def\s{{\bf s}}
\def\T{{\bf T}}
\def\U{{\bf U}}
\def\u{{\bf u}}
\def\V{{\bf V}}
\def\v{{\bf v}}
\def\W{{\bf W}}
\def\w{{\bf w}}
\def\X{{\bf X}}
\def\x{{\bf x}}
\def\Y{{\bf Y}}
\def\y{{\bf y}}
\def\Z{{\bf Z}}
\def\z{{\bf z}}
\def\0{{\bf 0}}
\def\1{{\bf 1}}
\def\RB{{\mathbb R}}

\newcommand{\red}[1]{{\color{red}#1}}

\title{CS 559: Your Project Title} % \\

\author{
% suggested author list, if you work with another student, provide names and email addresses for both.
Alice\textsuperscript{*} and Bob\textsuperscript{*}\\
\textsuperscript{*}{Stevens Institute of Technology}\\
alice@stevens.edu, bob@stevens.edu\\
Fall 2024
}
\date{\today}
\begin{document}
\maketitle
\begin{abstract}
Discuss the goals of this course project. Introduce the problem you are working on; Describe the potential impact of this problem; Summarize the results you have.
\end{abstract}

% Include the sections you need to present
\section{Introduction}
Describe the problem. Why is it important? Introduce context and motivating examples; State and summarize the scope and objectives of the project.
\section{Related Work}
Describe background. Introduce and cite what other people have done for this topic. Discuss the limitations of current approaches.

\section{Methodology}
Introduce the formulation of the problem (ideally with formulas and notations). Provide details of the methods you work on (use figures if needed).

\section{Experimental Setup}
Describe how you setup the experiments and the questions you try to answer using the experiments.
\subsection{Data}
Describe datasets you use including where you get the data, data statistics, etc. We encourage you to use public datasets.
\subsection{Evaluation Metrics}
Introduce evaluation metrics you use; If necessary, use equations.
\subsection{Comparison Methods}
List other methods you compare with (with citations) and the reasons you choose them.

\section{Results}
Analyze the results you get. Use tables or figures to show the results.
\begin{footnotesize}
\bibliographystyle{plainnat}
\bibliography{ref}
\end{footnotesize}%
\end{document}
