\documentclass{exam}

\usepackage{amsmath}

\usepackage{amssymb}

\usepackage{graphicx}

\usepackage{cite}
\usepackage{color} 
\usepackage{setspace}
\usepackage{hyperref}
\usepackage[linewidth=1pt]{mdframed}
\usepackage{tcolorbox}
\usepackage{hyperref}
\newcommand{\xx}{{\bf{x}}}
\newcommand{\yy}{{\bf{y}}}
\newcommand{\ww}{{\bf{w}}}

\pagestyle{headandfoot}
\runningheadrule
\firstpageheader{CS559: Machine Learning}{Name:        }{\textcolor{red}{Due: Oct. 17, 2024}}

\title{Assignment 3 Linear Classifier}
\date{}
\begin{document}
\maketitle
\thispagestyle{headandfoot}

\begin{center}
  {\fbox{\parbox{5.5in}{\centering
Homework assignments will be done individually: each student must hand in their own answers. Use of partial or entire solutions obtained from others or online is strictly prohibited. Electronic submission on Canvas is mandatory. }}}
\end{center}
\vspace{.5cm}

\underline{\bf Please follow the below instructions when you submit the assignment.}
\begin{itemize}
\item \textbf{Do not use any package/tool for implementing the algorithms; You can use packages for matrix/vector operations,  data processing,  or cross-validation}.
\item 
You shall submit a zip file named A3\_FirstName\_LastName.zip which contains:
\begin{itemize}
  \item a pdf file contains all your solutions for the written part
  \item python files (jupyter notebook or .py files)
\end{itemize}
\end{itemize}
\vspace{.5cm}

\begin{questions}

\question{\bf  Linear Discriminant Analysis} (5 points) Please download the ``processed.cleveland.data'' from \href{https://archive.ics.uci.edu/dataset/45/heart+disease}{Heart-disease data set} in the UCI Machine Learning repository and implement a binary Fisher's Linear Discriminant Analysis to distinguish no-heart disease (0) from heart disease(1 -- 4) and report your results.  Please read ``heart-disease.names'' for the explanation of features (13 features are used). Split data into training (80\%) and test (20\%). Write down each step of your solution.  You need to choose a decision boundary and classify the test samples based on the decision boundary you learned from the training data. Please report the data distributions (e.g., how many samples are no-heart disease and how many are heart disease). Then report your results on the accuracy, recall, precision, and F1 (assuming heart disease samples are positive samples) on the test data and plot the projected test samples using your learned w.
\vspace{5em}
 
 \question{\bf Generative methods vs Discriminative methods} (10 points) Please download the \href{https://archive.ics.uci.edu/ml/datasets/Breast+Cancer+Wisconsin+\%28Diagnostic\%29}{breast cancer data set} from UCI Machine Learning repository. You can either use ``breast-cancer-wisconsin.data'' or ``wdbc.data''. Please check their corresponding ``.names'' files for the explanation of features and labels.

\begin{enumerate}
\item (2 pts) Show that the derivative of the error function in Logistic Regression with respect to $\mathbf{w}$ is:
\begin{equation*}
\nabla_\ww E(\ww)=\sum_{n=1}^{N}(f(\xx_n)-y_n)\xx_n 
\end{equation*}

\item (4 pts) Implement a logistic regression classifier with maximum likelihood (ML) estimator using Stochastic gradient descent and Mini-Batch gradient descent algorithms. Divide the data into training and test. Choose a proper learning rate. Use cross-validation on the training data to choose the best model and report the recall, precision, and accuracy on malignant class prediction (class label malignant is positive) on the test data using the best model. Write down each step of your solution.

\item (4 pts) Implement a probabilistic generative model (the one in our lecture) for this problem. Use cross-validation on the training data and report the recall, precision, and accuracy on malignant class prediction (class label malignant is positive)  on the test data using the best model. Write down each step of your solution.
\end{enumerate}
\vspace{5em}


\question{\bf  Linear classification} (5 points) Please prove that 1) (2 pts) the multinomial naive Bayes classifier in log-space essentially translates to a linear classifier. 2) (3 pts) Logistic regression is a linear classifier .
\vspace{5em}




\end{questions}




\end{document}